\documentclass[12pt, a4paper, oneside]{article}
\usepackage{amsmath, amsthm, amssymb, bm, graphicx, hyperref, mathrsfs, algorithm}
\usepackage[noend]{algpseudocode}

\title{\textbf{Assignment\#4 CS201 Fall 2023}}
\author{Ben Chen(12212231)}
\date{\today}
\linespread{1.3}
\newcounter{problemname}
\newenvironment{problem}{\stepcounter{problemname}\par\noindent\textsc{Problem \arabic{problemname}. }}{\\\par}
\newenvironment{solution}{\par\noindent\textsc{Solution. }}{\\\par}
\newenvironment{note}{\par\noindent\textsc{Note of Problem \arabic{problemname}. }}{\\\par}

\begin{document}

\maketitle

\begin{problem}
    Prove that the principles of mathematical induction and well-ordering principles are equivalent.
\end{problem}

\begin{solution}
    \textbf{a)} First, we prove that the principle of mathematical induction implies the second principle.
    Suppose we have a proportinal function $F$ that satisfies the week induction.
    Let the propositional function $G(n) = F(1)\wedge F(2) \wedge \cdots \wedge F(n)$, from the week induction we have
    \begin{align*}
        G(n) = F(1)\wedge F(2) \wedge \cdots \wedge F(n) &\equiv F(n) \\
        F(n) \wedge (F(n) \rightarrow F(n+1)) &\equiv F(n+1) \\
        &\equiv G(n) \rightarrow F(n+1)
    \end{align*}
    and we have a tautology
    \begin{align*}
        G(n) \rightarrow G(n) &\equiv G(n) \rightarrow (G(n)\wedge F(n+1)) \\
        &\equiv G(n) \rightarrow G(n+1)
    \end{align*}
    So the function $G$ satisfies the week induction, that is $\forall k\ge 1.\ G(k)$
    \newline And we have $G(k) \rightarrow F(k)$ and $F(k) \rightarrow F(k+1)$ for all $k$, so
    \begin{align*}
        \forall k\ge 1.\ G(k) \rightarrow F(k+1) \equiv \forall k\ge 1.\ F(1)\wedge F(2) \wedge \cdots \wedge F(k) \rightarrow F(k+1)
    \end{align*}
    That is $F$ satisfies the second principle. So the principle of mathematical induction implies the second principle.
    \newline\indent Next, we prove that the second principle implies the principle of mathematical induction.
    Suppose we have a proportinal function $F$ that satisfies the second principle,
    \begin{align*}
        \forall k\ge 1.\ F(1)\wedge F(2) \wedge \cdots \wedge F(k) \rightarrow F(k+1) \equiv \forall k\ge 1.\ F(k) \rightarrow F(k+1)
    \end{align*}
    since the week induction has a weeker premise than the second. So $F$ satisfies the week induction.
    \newline  \newline\textbf{b)} Suppose that we have a non-empty set $S \subseteq \mathbb{N}$ which has no least element.
    Define a proposition function $F$ as $F(n) := n\notin S$.
    \newline \textbf{Base step:} $1\notin S$, so $F(1)$ is true.
    \newline \textbf{Inductive step:} Suppose that $F(1)\wedge F(2) \wedge \cdots \wedge F(n)$, that is for all $1\le k\le n$, $k$ is not in $S$.
    Since $S$ has no least element and $\forall 1\le k\le n.\ k\notin S$, $n+1$ cannot be in $S$, so $F(n+1)$ is true.
    \newline By principle of mathematical induction, we have $F(n)$ is true for all $n\in \mathbb{N}$, that is $S = \emptyset$, which is a contradiction.
    So the set $\mathbb{N}$ has a least element and the well-ordering principle holds.
\end{solution}

\begin{problem}
    Prove by induction that $A_1, A_2, \cdots, A_n$ and $B$ are sets
    \[ (A_1 - B)\cap (A_2 - B)\cap \cdots\cap (A_n - B) = (A_1\cap A_2\cap \cdots \cap A_n) - B\]
\end{problem}

\begin{solution}
    \textbf{Base step:} $n = 1$, $(A_1 - B) = (A_1 - B)$ is true obviously.
    \newline \textbf{Inductive step:} Suppose that \[(A_1 - B)\cap (A_2 - B)\cap \cdots\cap (A_n - B) = (A_1\cap A_2\cap \cdots \cap A_n) - B\]
    is true, then
    \begin{align*}
        &\quad(A_1 - B)\cap (A_2 - B)\cap \cdots\cap (A_n - B)\cap (A_{n+1} - B) \\
        &= ((A_1\cap A_2\cap \cdots \cap A_n) - B)\cap (A_{n+1} \cup B) \\
        &= ((A_1\cap A_2\cap \cdots \cap A_n) \cap \overline{B})\cap (A_{n+1} \cap \overline{B}) \\
        &= ((A_1\cap A_2\cap \cdots \cap A_n)\cap A_{n+1}) \cap \overline{B} 
    \end{align*}
    which is 
    \[ (A_1\cap A_2\cap \cdots \cap A_n\cap A_{n+1}) - B \]
    so the proposition holds for $n+1$ and by principle of mathematical induction, the proposition holds for all $n\in \mathbb{N}$.
\end{solution}

\begin{problem}
    Use induction to prove that if $p$ is a prime and $p | a_1a_2\cdots a_n$ where $a_i$ is integer then $p | a_i$ for some $i\in \{1,2,\cdots,n\}$.
\end{problem}

\begin{solution}
    \textbf{Base step:} $n = 1$, if $p | a_1$ is true, then $p | a_i$ is true for $i = 1$.
    \newline \textbf{Inductive step:} At first, we shall prove that if $p | ab$ and $p$ and $a$ are coprime then $p | b$ for prime $p$.
    From $gcd(a, p) = 1$ we have $ as + pt = 1$ and multiply both sides by $b$ we can obtain \[ abs + bpt = b\]
    Since $p | ab$ we have $p | abs$ and $p | bpt$, so \[p | abs + pbt = b\]
    Also, if for prime $p$ and $p | ab$ then $p | a$ or $p | b$ by exchanging the their roles.
    \newline \newline Suppose that for $n = k$, if $p | a_1a_2\cdots a_k$ then $p | a_i$ for some $i\in \{1,2,\cdots,k\}$.
    To prove for $n = k+1$, we have premise $p | a_1a_2\cdots a_{k+1}$. Let a number $b$ be
    \begin{align*}
        b := a_2a_3\cdots a_{k+1}
    \end{align*}
    so we have
    \[ p | a_1b\]
    if $p | a_1$ then the proposition holds, otherwise $p$ and $a_1$ are coprime, so $p | b$.
    And by hypothesis, since $b$ has $k$ factors, $p | a_i$ for some $i\in \{2,3,\cdots,k+1\}$.
    \newline\newline Thus, proof by cases, the proposition holds for $n = k+1$ and by principle of mathematical induction, the proposition holds for all $n\in \mathbb{N}$.
\end{solution}

\begin{problem}
    Let $P(n)$ be the statement that postage of $n$ cents can be formed using just 3-cent and 7-cent stamps. Prove that $P(n)$ is true for all $n\ge 12$.
\end{problem}

\begin{solution}
    \textbf{a)} $P(12)$ is true since $12 = 3\times 4$
    \newline $P(13)$ is true since $13 = 2\times 3 + 7$ 
    \newline $P(14)$ is true since $14 = 2\times 7$
    \newline\textbf{b)} $P(k)$ is true for $k = 12, 13, \cdots, n$ where $n \ge 14$.
    \newline\textbf{c)} Prove that $P(k+1)$ is true.
    \newline\textbf{d)} For $k + 1 = 15 + 3m$, $m\ge 0$, $P(k+1)$ is true since $k+1 = 12 + 3(m+1)= 3(m+5)$.
    \newline For $k + 1 = 16 + 3m$, $m\ge 0$, $P(k+1)$ is true since $k+1 = 13 + 3(m+1) = 3(m+3) + 7$.
    \newline For $k + 1 = 17 + 3m$, $m\ge 0$, $P(k+1)$ is true since $k+1 = 14 + 3(m+1) = 2\times 7 + 3(m+1)$.
    \newline\textbf{e)} By second principle of mathematical induction, $P(n)$ is true for all $n\ge 12$.
\end{solution}

\begin{problem}
    Describe a recursive algorithm for binary search.
\end{problem}

\begin{solution}
    \begin{algorithm}
        \caption{Binary Search}
        \begin{algorithmic}[1]
            \Procedure{BinarySearch}{$A, x, l, r$}
                \If{$l > r$}
                    \State \Return $-1$
                \EndIf
                \State $m \gets \lfloor (l+r)/2 \rfloor$
                \If{$A[m] = x$}
                    \State \Return $m$
                \ElsIf{$A[m] > x$}
                    \State \Return \Call{BinarySearch}{$A, x, l, m-1$}
                \Else
                    \State \Return \Call{BinarySearch}{$A, x, m+1, r$}
                \EndIf
            \EndProcedure
        \end{algorithmic}
    \end{algorithm}
\end{solution}

\begin{problem}
    Prove that the number of divisions requied by the Euclidean algorithm to compute the gcd of $a$ and $b$ is $O(\log b)$.
\end{problem}

\begin{solution}
    We shall prove for the \textit{i}th remainder $r_i$, we have
    \[ 2r_{i+2} < r_{i}\]
    To prove that, firstly consider the case when $r_{i+1} \le r_i / 2$, it holds definitely since $r_{i+2} \le r_{i+1} - 1$.
    So for the other case $r_{i+1} > r_{i}$, suppose that
    \[ r_i = qr_{i+1} + r_{i-2}\]
    $q$ cannot be bigger than $1$, so $q = 1$ for sure and $r_{i-2} < r_i - r_{i-1} < r_i / 2$.
    So, proof by cases, we have $ 2r_{i+2} < r_{i} $.
    \newline\newline When we iterate $r_1$ to $b$, say $r_{2n-1} = a\ \text{mod}\ b$, we got
    \begin{align*}
        b > 2r_{2n-2} > 4r_{2n-4} > \cdots > 2^{n}r_0
    \end{align*}
    where $r_0 = 1$ is the termination of algorithm. Thus, we can derive that 
    \begin{align*}
        b > 2^{n}r_0 = 2^{n}\ \Rightarrow\ n < \log_2 b
    \end{align*}
    And we have $2n = 2\log_2 b$ remainders. So the number of divisions required by the Euclidean algorithm to compute the gcd of $a$ and $b$ is $O(\log b)$.
\end{solution}

\begin{problem}
    Iterating the recurrence $T(n) = aT(n/2) + n$ yields $T(n) = \Theta(n)$ for $1\le a < 2$ and $T(1) \ge 0$.
\end{problem}

\begin{solution}
    Algebraically, we have
    \begin{align*}
        T(n) &= aT(n/2) + n = a(aT(n/4) + n/2) + n \\
        &= a^2T(n/4) + an/2 + n = a^2(aT(n/8) + n/4) + an/2 + n \\
        &\quad \vdots
    \end{align*}
    \begin{align*}
        &= a^i T(n/2^i) + a^{i-1}n/2^{i-1} + a^{i-2}n/2^{i-2} + \cdots + an/2 + n \\
        &\vdots \\
        &= a^{\log_2 n}T(1) + an/2^{1} + \cdots + an/2^{\log_2 n - 1} + n \\
        &= a^{\log_2 n}T(1) + n\sum_{i=0}^{\log_2 n - 1} \left(\frac{a}{2}\right)^i + n \\
        &= a^{\log_2 n}T(1) + n\left(\frac{1 - (a/2)^{\log_2 n - 1}}{1- a/2}\right) + n \\
        &= a^{\log_2 n}T(1) + n\left(\frac{2 - a/2 - (a/2)^{\log_2 n - 1}}{1- a/2}\right)
    \end{align*}
    since $a < 2$ we have $a/2 < 1$, so $(a/2)^{\log_2 n - 1} \rightarrow 0$ as $n \rightarrow \infty$ and $a^{\log_2 n} = n^{\log_2 a} < n$. Thus, we have
    $T(n) = \Theta(n)$ for $1\le a < 2$ and $T(1) \ge 0$.
\end{solution}

\begin{problem}
    Consider a deck of 52 cards. Answer the questions.
\end{problem}

\begin{solution}
    \textbf{a)} The number of full houses is
    \[\binom{13}{1}\binom{4}{3}\binom{12}{1}\binom{4}{2}\]
    \textbf{b)} The number of two pairs is
    \[\binom{13}{2}\binom{4}{2}\binom{4}{2}\binom{11}{1}\binom{4}{1}\]
    \textbf{c)} The number of flushes is
    \[\binom{13}{5}\binom{4}{1}\]
    \textbf{d)} The number of straights is
    \[\binom{10}{1}\binom{4}{1}^5\]
    \textbf{e)} The number of quadruples is
    \[\binom{13}{1}\binom{4}{4}\binom{12}{1}\binom{4}{1}\]
\end{solution}

\begin{problem}
    How many bit strings of length $8$ contain either four consecutive $0$s or four consecutive $1$s?
\end{problem}

\begin{solution}
    Since zero and one are symmetric, we only need to count the number of bit strings of either of them and multiply by two.
    The number of bit strings of length $8$ contain four consecutive $0$s is
    \[\binom{2}{1}^4 + \binom{2}{1}^3 + \binom{2}{1}^2 = 16 + 8 + 4 = 28\]
    since we have five slots to put four consecutive $0$s and the rest of them are $1$s or $0$s.
    But when we put four consecutive $0$s in the first or last slots, we can only put arbitary bits in the three of the rest.
    And when we put four consecutive $0$s in the second first or second last three slots, we can only put arbitary bits in the two of the rest.
    And when we put it in the middle, we can only put arbitary bits in the two of the rest.
    So the number of bit strings of length $8$ contain four consecutive $0$s is 56.
\end{solution}

\begin{problem}
    Prove that the following binomial is divisible by 2022.
    \[\binom{2020}{1010}\]
\end{problem}

\begin{solution}
    Firstly, 2022 can be factorized as $2\times 1011$ and $1011$ is a prime number. So we have
    \begin{align*}
        \binom{2020}{1010} &= \frac{2020!}{1010!1010!} \\
        &= \frac{2020\times 2019\times \cdots \times 1011\times 1010!}{1010!1010!} \\
        &= \frac{2020\times 2019\times \cdots \times 1011}{1010!} \\
        &= \frac{2\times 1010\times 2019\times 2018\times \cdots \times 1011}{1010!}
    \end{align*}
    \begin{align*}
        &= 2\times \frac{2019\times 2018\times \cdots \times 1011}{1009!} \\
        &= 2\times \frac{2019\times 2018\times \cdots \times 1011}{1009\times 1008\times \cdots \times 1} \\
        &= 2\times \frac{2019}{1009}\times \frac{2018}{1008}\times \cdots \times \frac{1011}{1} \\
        &= \boxed{2\times 1011} \times \left(\frac{2019}{1009}\times \frac{2018}{1008}\times \cdots \times \frac{1012}{2} \right)
    \end{align*}
    so $2022$ is a factor of the binomial.
\end{solution}

\begin{problem}
    Prove the hockey-stick identity by combinatorial argument.
\end{problem}

\begin{solution}
    Suppose that in a situation where we want to choose $r$ objects, for each $k$, we are choosing $r-k$ objects from $n+r+1$ objects as the first part and the rest to form the rest part
    skipping an object, then the number of ways to do the rest is
    \[ \binom{n+r+1 - (r-k+1)}{r - (r-k)} = \binom{n+k}{k}\]
    So summing from $k = 0$ to $k = r$ we have the number of choosing $r$ objects from $n+r+1$ objects which is
    \[ \binom{n+r+1}{r} = \sum_{k = 0}^r \binom{n+k}{k}\]
\end{solution}

\begin{problem}
    Solve the recurrence relation $a_n = 3a_{n-2} + 2a_{n-3}$ with $a_0 = 1$, $a_1 = -5$, and $a_2 = 0$.
\end{problem}

\begin{solution}
    The characteristic equation is
    \[ r^3 - 3r - 2 = 0 \]
    and the solution
    \[ r_1 = r_2 = -1\quad r_3 = 2\]
    so assume that
    \[ a_n = \alpha_1(-1)^n + \alpha_2n(-1)^n + \beta(2)^n\]
    we have
    \begin{align*}
        a_0 &= \alpha_1 + \beta = 1 \\
        a_1 &= -\alpha_1 - -\alpha_2 + 2\beta = -5 \\
        a_2 &= \alpha + 2\alpha_2 + 4\beta = 0
    \end{align*}
    so $\alpha_1 = 2$ and $\alpha_2 = 1$ and $\beta = -1$ and the solution is
    \[ a_n = 2(-1)^n + n(-1)^n - 2^n \]
\end{solution}

\begin{problem}
    Solve the non-homogeneous recurrence relations.
\end{problem}

\begin{solution}
    \textbf{a)} The characteristic equation is
    \[ r - 2 = 0 \]
    and the form of solution is
    \[a_n = \alpha_1 2^n + p(n)\]
    it's natural to assume that $p(n) = an^2 + bn + c$, so we have
    \begin{equation*}
        an^2 + bn + c = 2\left(a(n-1)^2 + b(n-1) + c\right) + n^2
    \end{equation*}
    \begin{equation*}
        (a+1)n^2 + (b - 5a)n + (c - 2b + 2a) = 0
    \end{equation*}
    So $a = -1$, $b = -5$ and $c = -8$ and all solutions are
    \[ a_n = \alpha 2^n-n^2 -5n - 8\]
    \textbf{b)} The initial condition $a_1  = 2$ gives that
    \[ \alpha = \frac{2+1+5+8}{2} = 8\]
    and the solution is
    \[a_n = 2^{n+3}-n^2-5n - 8\]
\end{solution}

\begin{problem}
    Use generating functions to solve the recurrence $a_n = 4a_{n-1} + 8^{n-1}$ with $a_0 = 0$.
\end{problem}

\begin{solution}
    Let $G(x)$ be the generating function of $a_n$, then
    \begin{align*}
        G(x) - a_0 &= \sum_{n=1}^{\infty} a_nx_n = \sum_{n=1}^{\infty} 4a_{n-1}x^n + 8^{n-1}x^n \\
        &= 4x\sum_{n=1}^{\infty} a_{n-1}x^{n-1} + x\sum_{n=1}^{\infty} 8^{n-1}x^{n-1} \\
        &= 4x\sum_{n=0}^{\infty} a_{n}x^{n} + x\sum_{n=0}^{\infty} 8^{n}x^{n} \\
        &= 4xG(x) + \frac{x}{1-8x}
    \end{align*}
    So we have
    \begin{align*}
        G(x) &= \frac{x}{(1-4x)(1-8x)} = \frac{1}{4(1-8x)} - \frac{1}{4(1-4x)} \\
        &= \frac{1}{4}\left(\sum_{n=0}^{\infty} 8^nx^n - \sum_{n=0}^{\infty} 4^nx^n\right) \\
        &= \sum_{n=0}^{\infty}\frac{1}{4} (8^n - 4^n)x^n
    \end{align*}
    Thus, the solution is
    \[ a_n = \frac{1}{4}(8^n - 4^n) \]
\end{solution}

\end{document}