\section{Related Work}
Gerlach et al.\cite{a-secure-risc} performed a systematic analysis of microarchitectural components in two commercially-off-the-shelf RISC-V processors T-Head C906 and SiFive U74. They discovered the available high-resolution timer, cache capacity and replacement policy and branch predictor specification in them.
With this in hand, the attacker can perform an attack on the Instruction Cache (ICache). If a victim has a secret-dependent branch, the attacker can jump directly to the two snippets of the if-else and measure the clocks difference in executing them. Since the executed instruction is present in ICache, it will be executed faster by the attacker. Based on the observation, the attacker is able to infer the secret bit. Similarly, it can be done without executing the victim's code. The attacker registers a fault handler with recording the end time, and maliciously sets the register bank that will trigger faults in victim's code, and finally jump to the code to immediately trigger the fault.
