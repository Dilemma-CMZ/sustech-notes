\documentclass{beamer}
\usepackage{listings}
\usepackage{color}
\usepackage{xcolor}
\definecolor{green}{rgb}{0,0.6,0}
\definecolor{gray}{rgb}{0.5,0.5,0.5}
\definecolor{mauve}{rgb}{0.58,0,0.82}
% init

\usetheme{Madrid}
\setbeamertemplate{itemize items}[triangle]
\setbeamertemplate{enumerate items}[default]
\lstset{
    frame=none,
    language=Java,
    showstringspaces=false,
    columns=fullflexible,
    basicstyle = \ttfamily\small,
    numbers=none,
    numberstyle=\tiny\color{gray},
    keywordstyle=\color{blue},
    commentstyle=\color{green},
    stringstyle=\color{mauve},
    breaklines=true,
    morekeywords={String,in,out,mv,cp,mkdir,rm,touch,vi},
    breakatwhitespace=true,
    tabsize=4
}
% appearance setup

\title{CS109 Lab 4}
\author{Ben Chen}
\institute{SUSTech}
\date{\today}

\begin{document}

\frame{\titlepage}

\begin{frame}[fragile]
    \frametitle{Loop - do while}
\textit{Do while} will execute the code at hand without checking the condition.
\newline
\newline
\begin{columns}
    \column{0.5\textwidth}
    \begin{itemize}
        \item check the condition at the end of each repetition
    \end{itemize}
    \column{0.5\textwidth}
    \begin{lstlisting}
do {
    inputNum = sc.nextInt();
    ...
} while(inputNum != guess);
    \end{lstlisting}
\end{columns}
\end{frame}

\begin{frame}[fragile]
    \frametitle{Compared to \textit{while}}
\begin{columns}
    \column{0.5\textwidth}
    \begin{itemize}
        \item less elegant
        \item redundant
    \end{itemize}
    \column{0.5\textwidth}
    \begin{lstlisting}
inputNum = sc.nextInt();

while(inputNum != guess) {
    ...
    inputNum = sc.nextInt();
}
    \end{lstlisting}
\end{columns}
\end{frame}

\begin{frame}[fragile]
    \frametitle{Loop - for}
\textit{For} is used in countable repetition, written as, 
\begin{lstlisting}
    for(<init>;<condition>;<step>) { // optional
        <code>
    }
\end{lstlisting}

And writing like this is also allowed,
\begin{lstlisting}
    for(;;) { // DON'T forget semi-colon
        <code> // never stop
    }
\end{lstlisting}
\end{frame}

\begin{frame}[fragile]
    \frametitle{Loop - for}
For example, output 9 times of the message
\begin{lstlisting}
    for(int i = 1; i < 10; i++) {
        System.out.println("This is for loop");
    } // counting from 1 to 9
\end{lstlisting}

\verb|i++| is executed after output, and then $i < 10$ will be checked;
\newline
\newline
Since when $i = 10$, condition is not satisfied, break the loop.
\end{frame}

\begin{frame}[fragile]
    \frametitle{Break and Continue}
    Break and continue both terminate the present period of loop.
\begin{itemize}
    \item \textbf{break} will end the loop
\end{itemize}
\begin{lstlisting}
    while(true) {
        ...
        if(i == 9) break;
    }
\end{lstlisting}
\begin{itemize}
    \item \textbf{continue} will execute the next period, skipping the rest.
\end{itemize}
\begin{lstlisting}
    for(int i = 1; i < 10; i++) {
        if(i == 5) continue;
        System.out.println("I am %d", i);
    } // skip "I am 5"
\end{lstlisting}
\end{frame}

\begin{frame}[fragile]
    \frametitle{Tips for OJ}
    To deal with multiple cases, you mustn't store the input in arrays.
\begin{lstlisting}
    int n = input.nextInt(); // number of cases
    for(int i = 0; i < n; i++) {
        <initialize variables>
        <input>
        ... // solve each single case
        <output>
    }
\end{lstlisting}
\end{frame}

\begin{frame}
    \frametitle{Switch Case}
    Another form of if-else, but notice that,
    \begin{itemize}
        \item \textbf{break} at the end of each Case
        \item Otherwise, other cases of code will be executed until \textbf{break}.
        \item \textbf{default} is not neccessary.
    \end{itemize}
\end{frame}

\begin{frame}
    \frametitle{Why we need Command Line?}
    At restaurant, the experienced with command line is like:
    \begin{itemize}
        \item Customer: I'd like a bottle of coke, hambergur and chip.
        \item Waiter: OK, right away!
    \end{itemize}
    With Graphic Interface:
    \begin{itemize}
        \item Customer: Menu please.
        \item Waiter: Here's the menu.
        \newline (minutes later)
        \item Customer: I want this(point to menu), this, and this.
        \item Waiter: OK, right away!
    \end{itemize}

\end{frame}

\begin{frame}
    \frametitle{Command Line Interface}
CLI is a software that renders your command.
\begin{itemize}
    \item In macOS\&Linux, it's Bash.
    \item In Windows, it's PowerShell or cmd.
\end{itemize}
When people talk about command line, people usually talk about Bash.
\begin{block}{Windows Subsystem for Linux}
    To install WSL, please follow \url{https://learn.microsoft.com/en-us/windows/wsl/install}
\end{block}
\end{frame}

\begin{frame}[fragile]
    \frametitle{Path}
Computer finds specific files by path.
\begin{itemize}
    \item Absolute path: Originating from root folder, for example, \textit{/Users/chenben/Desktop/CS/baby.java}
    \item Relative path: Originating from current folder, for example,\newline
    if I'm in \textit{/Users/chenben} , then, the path is \textit{Desktop/CS/baby.java}
    \item \verb|../| refers to parent folder, \verb|/| refers to root folder, \verb|~| refers to user folder.
\end{itemize}
Some commands about path,
\begin{itemize}
    \item \verb|pwd| checks the current path
    \item \verb|ls| shows files and folders in the current path
    \item \verb|cd| changes the current path
\end{itemize}
\end{frame}

\begin{frame}[fragile]
    \frametitle{Execute program}
There're three ways of executing program in command line,
\begin{itemize}
    \item \verb|./<program>| executes external program.
    \item \verb|<command>| executes internal program.
    If you added the path of your program to PATH, you can directly use the name of it as command.
    \item \verb|curl| executes remote program.
\end{itemize}
\end{frame}

\begin{frame}[fragile]
    \frametitle{Flag}
Flag is options for the command, present like \verb|-<flag>|.
\begin{lstlisting}[language=bash]
    ls -al # -al is a flag asking to show full list
\end{lstlisting}
Flag might need arguments, a full pattern of command is like, 
\begin{lstlisting}[language=bash]
    <command> arg1 arg2 -flag arg3 --Flag=arg4
    # arg1 and arg2 are required by command
    # arg3 is for -flag and arg4 can also be written like this.
\end{lstlisting}
\end{frame}

\begin{frame}[fragile]
    \frametitle{Basic Command}
You may need these commands,
\begin{lstlisting}[language=bash,morekeywords={mv,cp,mkdir,rm,touch,vim,man}]
    mv <file> <path>   # move file to path
    cp <file> <path>   # make a copy of file to path
    mkdir <name>       # create a folder
    rm <file>          # remove file
    rm -rf <folder>    # remove folder and its files
    touch <file>       # create a new file
    cat <file>         # show the content of file
    vim <file>          # modify the content of file
    man <cmd>          # show manual of command
\end{lstlisting}
\end{frame}
\end{document}
